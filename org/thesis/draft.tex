% Created 2023-11-20 Mon 11:32
% Intended LaTeX compiler: pdflatex
\documentclass[a4,11pt,twoside,final,hidelinks]{article}

\usepackage[utf8]{inputenc}
\renewcommand*\ttdefault{lcmtt}
\usepackage[T1]{fontenc}
\usepackage{graphicx}
\usepackage{longtable}
\usepackage{float}
\usepackage{wrapfig}
\usepackage{rotating}
\usepackage[normalem]{ulem}
\usepackage{amsmath}
\usepackage{amssymb}
\usepackage{capt-of}
\usepackage[font={small,it}]{caption}
\usepackage{geometry}
\usepackage[x11names,hyperref]{xcolor}
\usepackage{merriweather}
\usepackage[colorlinks=true,linkcolor=darkgray,citecolor=Tan4,filecolor=darkgray,urlcolor=darkgray]{hyperref}
\usepackage{multicol}
\usepackage{listings}
\usepackage{parskip}
\newcommand{\point}[1]{\noindent \textbf{#1}}
\usepackage{mathtools}
\DeclarePairedDelimiter\ceil{\lceil}{\rceil}
\DeclarePairedDelimiter\floor{\lfloor}{\rfloor}
\usepackage{indentfirst}
\setlength{\parindent}{1em}
\newgeometry{bottom=2in,top=2in,inner=1in,outer=1in}
\usepackage[font=itshape]{quoting}
\newenvironment{itquote} {\begin{quoting}\color{darkgray}} {\end{quoting}}
\usepackage{subfigure}
\setcounter{secnumdepth}{0} % sections are level 1
\author{Michael Ledger, Australian National University\thanks{u5582972@anu.edu.au}}
\date{November 20 2023}
\title{An Embedded Domain-Specific Language for Algebraic Tournament Structures\\\medskip
\large Construction and Evaluation of Sorting Structures in Haskell}
\hypersetup{
 pdfauthor={Michael Ledger, Australian National University},
 pdftitle={An Embedded Domain-Specific Language for Algebraic Tournament Structures},
 pdfkeywords={tournaments,edsl,haskell,fp,graphs,sorting,interactive,tui},
 pdfsubject={},
 pdfcreator={Emacs 29.1 (Org mode 9.7)}, 
 pdflang={English}}
\usepackage[style=ieee]{biblatex}
\addbibresource{draft.bib}
\begin{document}

\maketitle
\begin{abstract}
An Embedded Domain-Specific Language (commonly known as eDSL) is developed for
easy expression and analysis of arbitrary tournament structures. The core
tournament structure type is inspired by work on algebraic graph representations
in Haskell, and sorting networks, and enables composition in terms of either
interleaving or sequencing. The core design balances between allowing arbitrary
logic within a tournament structure and what static analysis can be achieved
given such a structure. A software library is developed that, upon future work,
can be capable of measuring characteristics such as length, fairness, resource
use, and reliability, on arbitrary tournament structures. Analogies from
tournaments to sorting networks are observed and common tournament structures
are contrasted with sorting networks to see how viable they can be as tournament
structures. A tool is also developed to perform and visualise tournaments
described within the eDSL.
\end{abstract}
\newpage

\newgeometry{margin=2cm}
\setcounter{tocdepth}{2}
\tableofcontents
\newpage
\newgeometry{top=2.5cm,bottom=2.5cm,left=4cm,right=2.5cm}

\definecolor{commentgreen}{RGB}{2,112,10}
\definecolor{eminence}{RGB}{108,48,130}
\definecolor{weborange}{RGB}{255,165,0}
\definecolor{frenchplum}{RGB}{129,20,83}

\lstset {
    language=haskell,
    frame=nil,
    tabsize=4,
    showstringspaces=false,
    numbers=left,
    %upquote=true,
    commentstyle=\color{commentgreen},
    keywordstyle=\color{eminence},
    stringstyle=\color{red},
    basicstyle=\large\ttfamily, % basic font setting
    emph={int,char,double,float,unsigned,void,bool},
    emphstyle={\color{blue}},
    escapechar=\&,
    % keyword highlighting
    classoffset=1, % starting new class
    otherkeywords={>,<,.,;,-,!,=,~,->,=>,::},
    morekeywords={>,<,.,;,-,!,=,~},
    keywordstyle=\color{weborange},
    classoffset=0,
}

\section{Introduction}
\label{sec:orga60ad6c}

Tournament structures are used to determine a ranking of players in a game, such
as a real-life sport or an eSport. Many different tournament structures are
employed in real games, for a variety of reasons; for example, what resources
are available, or how many participants there are, or how much time is
available. Moreover, the choice of tournament structure can have a significant
effect on the outcomes of the tournament. The choice of what tournament
structure is most appropriate for a given set of circumstance is mostly an open
one, and tournaments are ran as a combination of other tournament types.

There is possible contribution to be made in helping to determine what
tournament structures are most effective under the constraints of a real-world
tournament. This project aims to make such a contribution, in the form of a
software library that helps with the design and evaluation of tournament
structures.

More specifically, this project encapsulates my effort to do that over the
course of my undergraduate Advanced Computing Project (COMP4560), by creating a
Embedded Domain-Specific Language for tournament design and evaluation, as a
Haskell library. The software library developed for this project is available
online on \href{https://github.com/mikeplus64/journeyman}{ GitHub}\footnote{\url{https://github.com/mikeplus64/journeyman}}, which
also hosts its \href{https://mikeplus64.github.io/journeyman}{ API documentation}\footnote{\url{https://mikeplus64.github.io/journeyman}}. This thesis then serves as an overview of existing literature
into relevant topics, the software I have developed, and the results of using
this project to perform analysis on a variety of tournament structures.

\section{Acknowledgements}
\label{sec:org67ac692}

Much thanks is given to Ranald Clouston, who supervised this project through the
year.

\newpage

\section{Background and Motivation}
\label{sec:org8300067}

The economics of sports and eSports tournaments are staggering; in 2021 nearly
\$200bn USD was spent on sports betting \autocite{sportsbetworldwide} alone. At
least \$303m USD has been allocated in prize pools for \emph{DOTA 2} tournaments since
its release \autocite{biggestesportspools}. Quoting from an article The Economist
wrote in July 2020 about growth of eSports spurred on by the COVID-19 pandemic
\autocite{economistesportsgrowth}:

\begin{itquote}
Take “League of Legends”, perhaps the biggest e-sport in the world. It was
launched in 2009 by Riot Games, an American firm now owned by Tencent, China’s
biggest tech firm. It is a complex strategy game, in which teams of five players
command “heroes” in a battle to defeat each other. As many people play it
regularly as play tennis; at any one time, 8m people may be online. It also
supports a professional game that is, at least in terms of the number of players
earning a living from it, also larger than tennis. The final of the League of
Legends World Championship last year was watched live by 44m people. By
comparison the Super Bowl, America’s biggest live sporting event, was watched by
roughly twice that.

Twelve professional leagues now span all regions of the globe except Africa,
with 120 franchised teams and perhaps 1,000 professional players. Whereas tennis
stars in the world’s top 200 often struggle to make a living, “League of
Legends” players in America are guaranteed a minimum salary of \$75,000. There,
players are entitled to the same visas that other foreign athletes can get. The
average salary is closer to \$400,000, says Chris Greeley of Riot Games. Lee
Sang-hyeok, a Korean star, known by his tag “Faker”, may be the highest-paid
sportsman in his country.
\end{itquote}

With this in mind, it is worth asking how it is that players are ranked and
evaluated against eachother. That is: Are the tournaments reliable - are the
results repeatable? Are they fair? How could we determine this? Indeed, examples
exist of large real-world tournaments that were possibly unfair
\autocite{fifa-quant-unfairness}\autocite{fifa-risk-of-collusion} or where players
perceived that there would be a benefit to purposefully losing (throwing)
matches \autocite{Walker2012}\footnote{This example is quite poignant, including a scene
where both teams in a match of an Olympic badminton group stage endeavoured to
lose against eachother. A total of 4 teams were disqualified in this incident.
\autocite{double-elim-structure-efficacy-manipulation}} due to the structure of the
tournament.

\label{motivation} The initial motivation for this project was my experience in
helping to run tournaments for the games \emph{Diabotical} and \emph{Quake Champions} through
a \href{https://kuachi.gg}{ website}\footnote{\url{https://kuachi.gg}} I built for that purpose. Since its
inception in 2020, 79 tournaments have been completed through it. Frequently
questions of fairness in tournaments would arise due to some players seeming to
have ``unfairly'' hard or easy tournament brackets. This project then arises from
a desire to be able to quantify the fairness of the different tournament
structures that were employed -- if a software tool existed that could aide in
the design and evaluation of tournament structures (by, for instance, analysis
and/or simulation), that could have been employed to determine what tournament
structures were appropriate. Instead, the logic that describes tournaments on
\texttt{kuachi.gg} are hard-coded into the system, and while some composition options are
supported, it is a fairly inconvenient process to add new ones.

A key contribution this project seeks to make is to recommend tournament
structures that are resistant to ``false'' seeding (initial ranking). If a
tournament has a number of rank inversions in its seeding, does that number of
rank inversions reduce or increase after the tournament is ran? That is, to what
extent does the tournament really sort the players? This is a common problem
with real-world tournaments as often the relative skill level of players is
simply an unknown factor, and so guesses must be made.

This project contributes a small set of primitives that can be used to describe
arbitrary tournaments, an Embedded Domain-Specific Language (eDSL) for
tournament description that employs those primitives, an interpreter capable of
running and analysing tournament structures built using the eDSL. Additionally,
discussion of analysis of common tournament structures by the software library
is included in this thesis.

\subsection{\label{aims} Scope and aims of this project}
\label{sec:orgbbb65b6}
With this background in mind, these are the aims of the project:

\begin{enumerate}
\item To create a computer language that enables the convenient description of
common tournament structures, and also to enable creation of new tournament
structures as compositions of existing ones.

\item To create an interpreter for this language that can evaluate tournament
descriptions; to be able to run the tournaments described by the language
defined in Goal 1.

\item To systematically evaluate the efficacy of different tournament structures in
terms of quantifiable properties such as fairness, length, repeat-match
avoidance, and so on.
\end{enumerate}

I call the software library developed Journeyman.\footnote{A Journeyman is a term for
a consistently high-achieving career sportsperson that is not at the highest
level of their sport.}

\newpage

\section{Definitions}
\label{sec:org71be606}

As an aide to readers, the following glossary is provided. I specialise certain
terms to have exactly one specific definition in order to avoid confusion and
limit the scope of this thesis.

\begin{description}
\item[{Tournament}] A schedule of Matches by which a set of Players are ranked in
order of most-skilled first to least-skilled last.
\item[{Player}] A participant in a Tournament.
\item[{Match}] A pairing of exactly two Players in a tournament that has an
uncertain outcome to be determined by a game of some sort.
\item[{Bracket}] A schedule of games in an elimination tournament, commonly
expressed as a binary-tree diagram of player matchups.
\item[{Single Elimination}] A kind of elimination tournament in which the loser of
each match during a round is eliminated. Each round successively narrows the
field of remaining players until just one winning player remains. Also known
as a \uline{knockout} or \uline{sudden-death tournament}.
\item[{Double Elimination}] A kind of elimination tournament in which each player is
allowed to lose up to 1 match before they are eliminated. Also known as a
\uline{double knockout tournament}. The tournament is divided into two brackets; an
``upper'' or ``winner's'' bracket, and a ``lower'' or ``loser's'' bracket; the upper
bracket contains only players that have not lost a match up to that point, and
the lower bracket contains players who have lost a match in the upper bracket
prior to that point.
\item[{Seeding}] A ranking of the players of a tournament before the tournament
begins. Seeding can have a significant impact on the outcome of elimination
tournaments; consider the worst-case seeding of a Single Elimination
tournament in which the two best players match against each-other
immediately.

In a seeded tournament, Player \(0\) is the ``highest'' or ``best'' seed, denoting
the player with the highest skill. Player \((n - 1)\) is the ``lowest'' or ``worst''
seed, denoting the lowest skilled player.

\item[{Slaughter Seeding}] A method of seeding a tournament such that the initial
match for each player is the mirror of that player; the best player shall play
the worst, the second best shall play the second-worst, and so on.
\item[{Round Robin}] A kind of tournament where each player plays each other player
once.
\item[{Group Stage}] A kind of tournament where players are split into a fixed
number of groups, and a Round Robin tournament is played within each group.
those groups.
\item[{Stage}] Where a tournament is split into 2 or more sub-tournaments that run
in sequence, one after the other, those sub-tournaments are often coined
\emph{stages}.
\item[{Upset}] A match result is said to be an \emph{upset} if the higher seeded player
loses to the lower seed. (Regardless of how close their true skill levels may
be.)
\end{description}

\newpage

\section{Literature Review}
\label{sec:org01ece3e}

Before proceeding to the design and implementation of this project, I will
provide an overview of some of the relevant literature that informed how the
project proceeded and what gaps existed in knowledge that it could help narrow.
I consider literature across the topics of Domain-Specific Language design,
tournament design, and tournament fairness. Each sub-section here denotes a
particular paper that was considered.

\subsection{``Design Guidelines for Domain Specific Languages'' \citeauthor*{dsl-guidelines}\autocite{dsl-guidelines}}
\label{sec:orgbe0a12b}

This paper provides a list of guidelines to follow for the design of DSLs. I
followed it as a rough guide used in the process of designing the Journeyman
eDSL. The design guidelines are codified into a list of suggestions; here I
respond to each.

\begin{enumerate}
\item \textbf{``Identify language uses early''}
See \hyperref[aims]{\emph{Scope and aims of this project}}.

\item \textbf{``Ask questions''}
\begin{itemize}
\item \textbf{``Who is going to model in the DSL?''} Tournament designers who may find the
software useful.

\item \textbf{``Who is going to review the models?''} Tournament designers who may find
the software useful.

\item \textbf{``When?''} Prior to the design of a tournament, but when factors such as time
and resource constraints are available.

\item \textbf{``Who is using the models for which purpose?''} Tournament designers may use
the eDSL to help identify and analyse the efficacy of various tournament
structures, and apply those results; for myself to the real-world use-case
I identify in the \ref{motivation} section.
\end{itemize}

\item \textbf{``Make your language consistent.''} I apply an Algebra of Graphs-esque approach
to the design of the eDSL in order to keep its semantics simple.

\item \textbf{``Decide carefully whether to use graphical or textual realization''}

A textual representation will be the primary format for this eDSL, as it is
intended to be a programming library with the ability to describe arbitrary
tournament structures. That said, the software includes the ability to
visually interpret a tournament.

\item \textbf{``Compose existing languages where possible''}, and,
\item \textbf{``Reuse existing language definitions''} One of the key factors that decided the
choice of eDSL vs DSL was the ability for an eDSL to leverage its host
language to the full extent. Indeed, it does become apparent that many
tournament structures can be expressed quite conveniently with Haskell's list
functionality alone.

\item \textbf{``Reuse existing type systems''} As will be seen in later sections, the
type-system is used effectively here to restrict certain kinds of tournament
composition and to simplify the logic required to evaluate a given
tournament.
\end{enumerate}

\subsection{``Algebraic graphs with class (functional pearl)'' \citeauthor*{Mokhov2017Sep}\autocite{Mokhov2017Sep}}
\label{sec:org8acfb53}

This paper and the accompanying Haskell library \texttt{algebraic-graphs} largely largely
inspired the design of the core \texttt{Tournament} data type that the Journeyman eDSL
manipulates. The key observation is that graphs can be described using two
operators \texttt{Connect} and \texttt{Overlay} which each have desirable properties.

In the Journeyman eDSL, the \texttt{Overlay} operator is reproduced to mean the
interleaving or parallel existence of two sub-tournaments, and can be intuited
as an algebraic sum operator. The \texttt{Connect} operator is not reproduced, as it does
not have an intuitive analogue in tournament structures; instead, a \texttt{Sequence}
operator connects tournaments by running them one after the other, and can be
viewed as an algebraic product operator. Type information is then used to
guarantee that we can turn any arbitrary tournament, which can nest combinations
of \texttt{Overlay} and \texttt{Connect} at will, into a ``flattened'' stream of rounds. For more
information, see \hyperref[design]{\emph{Design}}.

\subsection{``The structure, efficacy, and manipulation of Double Elimination tournaments'' \citeauthor*{double-elim-structure-efficacy-manipulation}\autocite{double-elim-structure-efficacy-manipulation}}
\label{sec:org5d6e85a}

This paper provides information about Double Elimination tournaments in
particular. Several important theorems are provided for their design - in
particular a result about the optimal linking between the upper and lower
brackets of a Double Elimination tournament - as well as statistical analysis
performed on the efficacy of tournaments.

Statistical analysis is performed to compare the reliability of
Single Elimination tournaments to Double Elimination ones, where
Double Elimination is shown to be much more efficacious in allowing the most
skilled player to win. Simulations are performed using chosen models for the
probabilities of players winning against each other, rather than on real-world
data.

Manipulation of Double Elimination tournaments is also considered, and an
interesting case study provided to demonstrate the need for tournaments that are
robust against manipulation: ``in the 2012 Olympics, four of the top badminton
teams were disqualified for trying to intentionally lose matches, causing an
uproar and angering fans. While the tournament structure used there was not a
DET, this demonstrates that players really will exploit poor tournament design
when possible.'' The importance of seeding in the outcome of elimination
tournaments is noted as well. Several theorems are provided on the complexity of
manipulation of a tournament by players.

Double Elimination \textbf{Link Functions} are described in this paper: The Link Function
is the algorithm that chooses where in the lower bracket a player from the upper
bracket should go after a loss. The choice of Link Function is quite important
in order to avoid re-matching players who already faced each other in the upper
bracket as much as possible \footnote{Avoiding rematches has intuitive benefits;
players are able to gauge their skill against multiple opponents, and thus so
does the tournament.}. Two operations, named Swap and Reverse, are described in
constructing a Link Function, and an optimal Link Function that avoids rematches
as much as possible can be constructed using \(\ceil{\log(R)}\) Link or Swap
operations.

\subsection{``Double Elimination Tournaments: Counting and Calculating'' \citeauthor*{double-elim-cc}\autocite{double-elim-cc}}
\label{sec:org3065fe3}

This paper provides broad information about the construction of Double
Elimination tournaments. The efficacy of ``unbalanced'' Double Elimination
tournaments is considered in detail. A system for uniquely numbering
Single Elimination tournaments is also provided, with extension then to number
Double Elimination tournaments by the structure of the lower bracket as well as
the linking function used.

Statistical analysis is performed by using an assumed preference matrix,
denoting the pairwise probabilities of one team winning a game against another.
Using a preference matrix contrasts against methods used in ``real'' games to
calculate the probability of a player winning a particular match against
another, such as the Elo Rating System\autocite{EloDesc}; in Elo, all players are
assumed to have an absolute quantifiable skill level, that satisfies
transitivity; if player \(A\) is more skilled than player \(B\), and player \(B\) is
more skilled than player \(C\), then \(A\) must be more skilled than player \(C\). A
preference matrix approach allows for the fact that some players may do
particularly well or poorly against other players. It may be possible to
calculate a preference matrix from existing public data from existing games, by
assigning a secondary ranking to players by treating each possible pair as its
own separate game.

The larger Double Elimination tournament shown in this paper does not to have a
``balanced'' lower bracket. Convention in modern Double Elimination tournaments is
that, to maximise fairness and minimise the number of rounds required, one
should alternate between rounds where players are from the lower bracket play
against each other, and where ``new'' players are added in to the lower bracket
from a round in the upper bracket. This is shown in ``The structure, efficacy,
and manipulation of Double Elimination tournaments''
\autocite{double-elim-structure-efficacy-manipulation}.

\subsection{``Simulating competitiveness and precision in a tournament structure: a reaper tournament'' \citeauthor*{reaper}\autocite{reaper} and ``Reaper Tournament System'' \citeauthor*{reaper2017}\autocite{reaper2017}}
\label{sec:orgdb9c39b}
This pair of papers describes a novel tournament structure coined ``Reaper
tournaments''. \autocite{reaper2017} describes most of the results and
\autocite{reaper} develops the knowledge of the Reaper tournament system further,
and creates a similar (but new) tournament structure called \emph{Reaper elimination}.

The structure of a Reaper tournament is that it operates initially as an
inverted Single Elimination tournament, where only the losers of each match
advance along the Single Elimination bracket. An algorithm is applied to
successively select and eliminate the worst player successively. Indeed, Reaper
tournaments seem to be analogous to common Selection Sort algorithms; and by
only eliminating a single player in each round, it is able to achieve 100\%
ranking precision - or at least that no player shares the same result as
another.

The number of matches required in a Reaper tournament is not given a general
formula in the system, nor the number of rounds, which is a significant weakness
to its adoption as a tournament structure in practice - tournaments need to
happen usually within some known time constraints. Description of the Reaper
tournament system as a sorting network, though the Journeyman eDSL, may help to
elucidate its properties. For \(n=8\), the Reaper tournament requires \(m\in[15,17]\)
matches compared to \(m=14\) for double elimination or \(m=28\) for a round-robin.

It is also shown that the \emph{stability progression}, measuring whether winning a
game is more desirable than losing, is preserved in the Reaper tournament
structure. It is never a desirable outcome to lose a match in the Reaper
tournament structure.

Additionally, two-stage tournament systems where a Group Stage precedes an
Elimination are considered \autocite{reaper}. The group stage has multiple groups
of players in each group, and a tournament structure such as round-robin (or
Reaper), is conducted within that group. Such structures are quite common in
practice.

Various metrics are created to measure the efficacy of tournaments in practise
and in simulation. The key metrics are \emph{ENM}, meaning ``expected number of
matches'', \emph{ARW}, the ``average rank of the tournament winner'' (ideally, 1), and
\uline{RankCor} \(\in [0,1]\) where a value of 0 means the tournament had a completely
random result with respect to the player's ``true'' skills/rankings, and a value
of 1 means that the tournament perfectly preserved those a priori rankings.

Theoretical experiments on 8 player tournaments are conducted that show the
excellent RankCor of the original Reaper tournament structure. Double
Elimination stages are also shown to have quite good RankCor (at this size of
tournament). Real-world tournament data is used that demonstrates the robustness
of Double Elimination tournaments in terms of \uline{RankCor}, with Reaper tournaments
also performing excellently, though doubling the number of matches required.

\subsection{Description of the Reaper tournament structure algorithm}
\label{sec:org33cecd6}
I reproduce in my own words the algorithm for the Reaper tournament structure
here.

Information:
\begin{itemize}
\item Each player has a \uline{respect list} of players who they have previously lost to.
This is updated every time a game occurs.
\item The tournament is assumed to be \(n=2^k\) in size; there must be a power-of-2
number of players.
\end{itemize}

Steps:
\begin{enumerate}
\item \uline{Reaper selection}: In Round 1, pairs of players are matched together, so that
every player is in a match. The losers in the round are then paired against
each other, and again, until a round where only a single player loses a match
(who lost all matches prior to this round), and they are eliminated from the
tournament. Let the winner of the final game in this step be the \uline{Reaper}.

This basically describes an ``inverted'' Single Elimination tournament - where
to proceed to the next round, you must \uline{lose} the current round. The ``winner''
(i.e., loser of all games) then of this inverted Single Elimination
tournament is the one who is actually eliminated from the tournament.

The question of what matching algorithm is used is left open by the authors
of the paper, but it is likely significant in determining the outcome of the
\uline{Reaper selection} stage.

\item \uline{Reaper candidates}: A \uline{candidate list} is created consisting of:
\begin{itemize}
\item If there are players who are not in a respect list, those players.
\item Otherwise, the players who are in the respect list of the Reaper.
\end{itemize}

The size of the candidate list then determines the next step:
\begin{itemize}
\item If \(> 1\), proceed to (3).
\item If \(= 1\), proceed to (4).
\item Otherwise (\(= 0\)), the tournament ends.
\end{itemize}

\item \uline{Candidates match}: The two best players play each other. Update the respect
lists accordingly and go back to step (2).

\item \uline{Reaper match}: The single player in the candidates list plays the Reaper. The
loser here is eliminated and is ranked above the previously eliminated
participant, while the winner is set to be the new Reaper.
\end{enumerate}

The expression of this structure in an eDSL is challenging, as the tournament
needs to be able to ``respond'' to the results of matches in order to maintain a
respect list and candidates list.

\subsection{Reaper Elimination}
\label{sec:orgfca456e}
A follow-up structure is proposed in the second paper \autocite{reaper2017} that
develops the Reaper tournament structure to give it an upper bound on the number
of matches required, and a static tree structure. Thus, it is likely a
tournament that could be expressed as a sorting network. It is shown that the
number of matches required is \(O(N\log_2 N)\).

\subsection{``Quantifying the unfairness of the 2018 FIFO World Cup qualification'' \citeauthor*{fifa-quant-unfairness}\autocite{fifa-quant-unfairness} and ``Risk of Collusion: Will Groups of 3 Ruin the FIFA World Cup?'' \citeauthor*{fifa-risk-of-collusion}\autocite{fifa-risk-of-collusion}}
\label{sec:org5c18710}

These papers look at real-world sports tournaments, namely the FIFA series of
soccer/football tournaments. As these are huge events with massive prize pools
that carry great prestige for participating teams, nations, and hosts,
examination of these events for fairness criteria is important. These papers
demonstrate how real-world data can be used to examine and quantify fairness of
tournament structures.

It is shown in \autocite{fifa-quant-unfairness} that the origin continent of a team
had an out-sized effect on the likelihood of a team in qualifying into the FIFA
World Cup in 2018. It is found that a fixed draw rather than a random draw for
qualification would reduce unfairness.

Unfairness is measured by ``ranking the teams according to their Elo, and summing
the differences of qualifying probabilities that do not fall into line with this
ranking''.

In \autocite{fifa-risk-of-collusion}, the conditions required to aggravate the risk
of collusion between teams is examined. This can occur when two teams in a Group
Stage are already guaranteed entry into the proceeding stage, but the result of
their match can adversely affect whether or not another team in that group makes
it through to the next stage or not. Examples of collusion are examined in
real-world games. Games such as soccer where a draw is a possible outcome may be
susceptible to colluding outcomes; teams can agree in advance to draw against
each other, and neither will lose face nor prestige, while still possibly being
able to gain the points required to proceed on to the next stage of the
tournament.

\subsection{``Handling fairness issues in time-relaxed tournaments with availability constraints'' \citeauthor*{fairness-time-relaxed}\autocite{fairness-time-relaxed}}
\label{sec:orgfbfaddb}

This paper examines computational complexity of time-relaxed tournament game
scheduling. That is, the problem of scheduling games where there is not a tight
deadline to complete the games, but there may be sporadic player and venue
availability. This situation frequently occurs during ``long format'' group stage
formats which are ran over weeks or months, where the scheduling of each game is
done by each player participating in that game together. However, this is out of
scope to the research aims of this project. The fairness measures proposed by
this paper also concern scheduling, which is outside of the scope of this
project.

\subsection{``The impossibility of a perfect tournament'' \citeauthor*{perfect-impossible}\autocite{perfect-impossible}}
\label{sec:org59ae875}

This paper provides an important result that shows that their constructed
fairness and balance metrics trade off against one-another, and elimination
tournaments cannot be constructed that maximise both metrics. The author
concludes that a perfect tournament design cannot be made because of the
inherent uncertainty of outcomes and player seeding; indeed, if perfect ranking
was already available at the outset, there would be little point to running a
tournament in the first place. The author also provides discussion on the
tournament outcomes and spectator interest; where players who play optimally are
perceived to be dull or unimaginative.

The \emph{fairness} metric here is that the sum of the ranks of winners of each match
must be maximised across the whole tournament. This is an interesting definition
that intuitively works quite well when the tournament structure is also
minimising the number of matches required - one could construct a
degenerate-case tournament structure that maximises this sum, by, for example,
matching 2 weak players repeatedly until the sum generated by the winners of
those matches must be greater than the sum generated by the winners of the other
matches in the tournament.

The \emph{balance} metric here is to minimise the difference in ranks between players
across all matches. By doing this, you create tournament structures that provide
as more information about players who are closely matched. In the single
elimination case, it is clear that maximising balance minimises fairness.
Maximising balance can have the effect of increasing spectator interest, as
closer games are assumed to be more exciting to watch than ``blow-out'' games,
which I can validate from my own anecdotal experience.

\subsection{``A new knockout tournament seeding method and its axiomatic justification'' \citeauthor*{knockout-seeding}\autocite{knockout-seeding}}
\label{sec:org8066523}

This paper demonstrates the determining effect of seeding to Single Elimination
tournament outcomes, and proposes an ``equal gap'' seeding method contrary to the
traditional ``slaughter seeding'' method, that, under a deterministic domain
assumption, satisfies the fairness, competitive integrity, and equal rank
difference axioms that are introduced. Here, determinism refers to assuming that
given any match, the player with the highest seed/skill shall win.

Applicability of the proposed seeding method outside of the domain assumption is
an open question, and may be a useful application of the Journeyman library to
examine its effects.

\subsection{``The efficacy of tournament designs'' \citeauthor*{tournament-efficacy}\autocite{tournament-efficacy}}
\label{sec:org5946bc9}

Efficacy of tournaments is analysed in terms of ranking inversions exhibited at
the end of the tournament. A valuable result is that triple-elimination does not
greatly improve the efficacy of ranking players compared to Double Elimination,
especially when accounting for the extra matches required.

Swiss-style tournaments are shown to be very effective at ranking players and
generally exhibit fewer inversions than any other format considered, for the
same number of matches - although Swiss-style tournaments use a matching
algorithm each round to determine who plays who, they are ran to a fixed number
of rounds, so they can be engineered to desired level of accuracy and matches.
Swiss-style tournaments are shown to be superior to single/double/triple
elimination and group stage tournaments. The choice of matching algorithm here
likely has the greatest effect on result.

\newpage

\section{\label{design} Design}
\label{sec:orgfc3ec97}

In this chapter I will explain the design of the software artefact developed. To
understand the final design that was implemented, it's first worth examining
what information lead to it.

\subsection{Analogies between sorting networks and tournaments}
\label{sec:org3a34125}

A key observation of made at the outset of this project is that there is an
analogy between sorting networks and many tournament structures. Here, a sorting
network refers to a fixed schedule or ``network'' of comparisons between a fixed
set of objects. Each comparison has a fixed coordinate or \emph{wire} in the network,
and results in either the objects staying in the same position as they were (if
they were already in order with respect to eachother), or they exchange places
if not.

I call a sorting network ``partial'' if it does not determine a complete ordering
among players; for instance if out of 16 elements it determines the greatest 8
in order, but leaves the remaining 8 in two buckets of 4 items where only the
order of the buckets is known.

\begin{figure}[H]
\hfill
\subfigure[\label{fig:se8}Progression of a Single Elimination tournament under Slaughter Seeding]{\includegraphics[width=0.4\textwidth]{./single-elim-bt8.mermaid.pdf}}
\hfill
\subfigure[Progression of the same Single Elimination tournament, expressed as a sorting network]{\includegraphics[width=0.4\textwidth]{./singleelimnet8.png}}
\hfill
\end{figure}

A Single Elimination tournament has a partial sorting network construction. This
is because at each round we can draw matches between the winners of the previous
round only; that is, those players that now occupy the ``high'' position of the
sorting network, after a single step of the network. At the same time, the
losers of each round are simply not given any further comparisons (i.e.,
matches) from the point that they were eliminated, and so remain ranked at
whatever point they were eliminated.

Similarly, we can also construct a Double-Elimination tournament by sorting
network, by creating matches between those players that lost in first round of
the ``upper'' bracket, and then alternating rounds that either accept new losers
from the previous upper bracket round into new matches in the lower bracket, or
that play off players who are in the lower bracket.

Swiss-style tournaments may fall under sorting networks if a pairing algorithm
is chosen that does not enforce a no-rematch rule. Indeed, the sorting algorithm
titled sorting algorithm \emph{I Can't Believe It Can Sort}
\autocite{cant-believe-it-can-sort}, a mistaken version of insertion or bubble
sort, can be viewed as a Swiss-style tournament with a matching algorithm that
allows rematches, ran to \(\Theta(n^2)\) rounds for \(n\) players; players that are
closely ranked are repeatedly matched/compared. With the result of
\citeauthor*{tournament-efficacy} in mind, this sorting network with comically
poor characteristics may indeed be an extremely effective tournament structure
-- if a quadratic number of rounds is permissible.

\begin{figure}[htbp]
\centering
\includegraphics[width=.9\linewidth]{./cantbelieve8.png}
\caption{\label{fig:cantbelieve8}The first 8 rounds of an \uline{I Can't Believe It Can Sort} tournament structure, visualised in the Journeyman-UI tool.}
\end{figure}

I call tournaments that have a direct sorting network analogy Sorting Network
Tournaments (SNT). SNT does not encapsulate all tournaments that intuitively
have a static structure. Namely, they cannot express round-robin tournaments. In
a round-robin tournament, every player has exactly one match against every other
player. But after a single round in a SNT, which players occupy what slots of
the network are unknowable; hence we cannot proceed after a round of a SNT to
pair ``every other'' player without the possibility of a rematch. Since in a SNT
the players must exchange position upon a player from a numerically greater slot
beats a player from a numerically lesser slot, a SNT Round-Robin requires
knowledge of each prior round to avoid rematching.

We can work around this by defining another category of tournaments that
intuitively have a static structure (given some \(n\) number of players), which I
call Static Non-Sorting Network Tournaments (SNSNT), in which swap-exchange
matches are removed, but a sorting mechanism is available based on points
accumulated by players each match over a fixed number of rounds. A round-robin
is such a tournament, since we can trivially construct a full round-robin
schedule in advance using known algorithms such as the Circle Method.

\newpage
\subsection{\label{design} A Type for Tournaments}
\label{sec:org06f86fd}

The core type of the eDSL is described here. The primary feature of this type is
its simplicity and ease of analysis. Tournaments in this library depend on two
key operators inspired by the \texttt{algebraic-graphs} package, which are used to
manipulate a small set of primitive tournament types. The most basic primitive
chosen is \texttt{Match :: (Int, Int) -> Tournament}, which simply requests that match
specified by the pair of sorting network slots. To justify the design of the
core eDSL type, we shall build it from scratch:

Firstly, we need a way to, at a minimum, have two matches run at once. We can
then generalise this operatorto allow any two tournaments to run in lock-step.
Thus, the first operator defined is named \texttt{Overlay}, which overlays two
tournaments together, running them in parallel. The \texttt{(+++)} operator provides an
infix shorthand for this operator. For instance, the first round of a 4-player
Single Elimination tournament can be encoded with just \texttt{(+++)} and \texttt{Match}:
\texttt{(Match 0 3) +++ (Match 1 2)}, or 8-player: \texttt{(Match 0 7) +++ (Match 1 6) +++
(Match 2 5) +++ (Match 3 4)}. Note that unlike in a binary tree representation,
the order of matches with respect to each-other has no effect on what tournament
is described. To illustrate this, recall Figure \ref{fig:se8}; how would the expected
outcome change if at the initial stage, the match \((0,7)\) was swapped with
\((2,5)\)?

The \texttt{(+++)} operator satisfies these algebraic properties:
\begin{enumerate}
\item Transitivity: \texttt{(x +++ y)} \(\equiv\) \texttt{(y +++ x)}
\item Associativity: \texttt{(x +++ (y +++ z))} \(\equiv\) \texttt{((y +++ z) +++ x)}
\item Identity: \texttt{(Empty +++ x)} \(\equiv\) \texttt{(x +++ Empty)} \(\equiv\) \texttt{x}
\end{enumerate}

Secondly, we need a way to, at minimum, run two matches one after the other. We
can then generalise this operator to allow any two tournaments to run in
sequence. Thus, the second operator defined is \texttt{Sequence}, which connects two
tournaments together by running them one after the other. It is given an infix
shorthand \texttt{(***)}. It cannot satisfy transitivity, it does satisfy
associativity. It does also \emph{not} satisfy Identity; the reason for this is that
creating empty rounds may be useful in the context of certain kinds of
interleaving two tournaments together, such as allowing one tournament to run in
even rounds, and another to run in odd ones. This could be achieved by
sequencing an \texttt{Empty} round to every other round in the two input tournaments,
and then overlaying them both -- but only if the Identity property is not
granted to \texttt{Sequence}.

We now have a simple tournament type \texttt{data Tournament0 = Overlay Tournament0
Tournament0 | Sequence Tournament0 Tournament0 | Match Int Int | Empty} which I
claim is isomorphic to SNT, under an interpretation of \texttt{Match} always causing a
swap-exchange to occur if there was an upset result. This structure does present
some challenges in being able to generate a \uline{schedule} of matches from it; since
there can be an arbitrary, possibly unbalanced, mix of \texttt{Overlay} and
\texttt{Sequences}; consider the following tournament:

\begin{figure}[htbp]
\centering
\includegraphics[width=.9\linewidth]{./unbalanced.mermaid.pdf}
\caption{\label{unbalancedt0}An unbalanced combination of sequences and overlays in a tournament}
\end{figure}

\begin{lstlisting}[language=haskell,numbers=none]
unbalancedT0 = ((Match 0 1) +++ (Match 2 3))
           +++ ((Match 4 5) *** (Match 0 1))
\end{lstlisting}

How to interpret this may not be immediately obvious. Obviously, on the second
branch, match \((4,5)\) and \((0,1)\) are intended to run one after the other. But
what does it mean to overlay on this structure? The interpretation taken by
Journeyman is logically to insert an equivalent number of ``empty sequences'' that
balance the tournament with the side that has more sequences than the other.
That is, we \emph{aligning} the two sides of an overlay operation.

\begin{figure}[H]
\hfill
\hspace*{-2cm}\subfigure[\label{fig:balance1}Step 1]{\includegraphics[width=9cm]{./unbalanced2.pdf}}
\hfill
\subfigure[\label{fig:balance2}Step 2]{\includegraphics[width=6cm]{./unbalanced3.pdf}}
\hfill
\end{figure}

A tournament that is expressed as a sequence of overlays of matches is said to
be in Tournament Round Normal Form (TRNF); it can be thought of as a sequence of
\uline{rounds} of a tournament or sorting network, and it is now trivial to traverse
this to create a schedule of rounds, of matches. Tournaments that satisfy this
property are much easier to interpret than tournaments that don't.

By encoding a measure of \uline{depth} (think, number of rounds) of a tournament into
its type, we can greatly simplify functions that interpret tournaments; so long
as we can transform any tournament into TRNF, we only need to express a function
in terms of a single round of a tournament to be able to lift it to work on
arbitrarily-deep tournaments.

\newpage
Consider if we simply add a type-level natural number to the \texttt{Tournament} type:

\begin{lstlisting}[language=haskell,numbers=none]
data Tournament :: Nat → Type where
  -- | Connect two tournaments by having them occur
  -- simultaneously; analogous to the "Overlay" operator
  -- from algebraic-graphs
  Overlay :: Tournament a → Tournament a → Tournament a
  -- | Connect two tournaments by running one after the
  -- other. Note the change in depth
  Sequence :: Tournament a → Tournament b → Tournament (a + b)
  -- | Request a single match to be played
  One :: Match → Tournament 1
  -- | Do nothing
  Empty :: Tournament t

-- | Play the two players occupying the /slots/ specified by
-- this match together
data Match = Match Int Int
\end{lstlisting}

This is now getting quite close to the final type used by the Journeyman eDSL.
Under this design, so long as we have a function that can lift a function from
rounds to sequences of rounds as above, and a function that can transform an
arbitrarily-deep/unbalanced tournament into TRNF, then interpreters over
tournaments effectively only need to be concerned with singular rounds. As it
turns out, adding this type parameter guarantees that our function \texttt{toTRNF ::
Tournament a -> NonEmptyList (Tournament 1)} does indeed only return single rounds.

Since we only care about distinguishing tournaments of depth \(> 1\) between
tournaments of depth \(\le 1\), we can swap out type-level arithmetic for a more
lossy representation of \(\mathbb{N}\): \texttt{data Depth = TOne | TMany}, and
parameterise the tournament over that instead.

We also need an operation to allow a tournament to be parameterised over the
number of players available. We would also like to be able to manipulate that
player count, and moreover, to have a convenient mechanism for dividing
tournaments along the ``player'' axis, to, for instance, represent a group stage
round robin format.

\begin{lstlisting}[language=haskell,numbers=none]
  ByPlayerCount :: (Int -> Tournament t) -> Tournament t
  ByFocus :: (PlayerCount -> [Focus]) -> Tournament t

-- | A focus represents a slice over the current set of
-- players. Tournament interpreters must use the current
-- focus as an offset for matches generated within.
data Focus = Focus { start, length :: Int }
\end{lstlisting}

We also lack a way to tell a tournament interpreter what sorting strategy to
use; should every upset match result in a swap-exchange? Or should no swaps
occur, and only a final tally of points be used to finalise player standings?
Additionally, we can not yet support tournaments that require past match results
to decide what to do (e.g., conventional sorting algorithms). This motivates the
final three constructors:

\begin{lstlisting}[language=haskell,numbers=none]
  BySwaps :: Tournament t -> Tournament t
  ByPointAward :: Tournament t -> Tournament t
  ByStandings :: (Standings -> Tournament t) -> Tournament t
\end{lstlisting}

For an API-centric view of the \texttt{Tournament} type, see the
\href{https://mikeplus64.github.io/journeyman/Tourney-Algebra-Unified.html}{ \texttt{Tourney.Algebra.Unified}}\footnote{\url{https://mikeplus64.github.io/journeyman/Tourney-Algebra-Unified.html}} module. The main conceptual difference is that by
adding a \texttt{ByStandings} constructor, tournaments can no longer necessarily be
inspected completely in advance; they may have arbitrarily many rounds,
conditional on certain match results. Therefore, an abstraction is built that
allows tournaments to be inspected to their first ``pure'' subset -- that is, the
first (in terms of \texttt{Sequence} order) sub-tournament before a call to
\texttt{ByStandings}.

The design of an abstract streaming type to support this follows from other
open-source streaming libraries for Haskell, namely, \href{https://hackage.haskell.org/package/streaming}{\texttt{streaming}}. The streams
defined in Journeyman however support \(O(1)\) concatenation and, as above, have
the key difference that they do not necessarily require any effectful code to be
ran to be able to inspect an element of the stream; a stream can purely yield
values (i.e., matches, or rounds) that can also be popped off purely.
Additionally, operators are defined for \emph{aligning} two streams of values
together, which is how the operation in Figure \ref{fig:balance1} is implemented.

With this representation, compilation of tournaments can ultimately reduces them
to a single stream of a very small set of commands, which become trivial to
interpret: \texttt{data Op = BEGIN\_ROUND | END\_ROUND | MATCH Match | PERFORM\_SORTING
Focus SortMethod}. The key detail is that an interpreter must not proceed past
an \texttt{END\_ROUND} or a \texttt{PERFORM\_SORTING} until all the relevant matches have been
completed\footnote{For \texttt{END\_ROUND}, all matches are relevant. For \texttt{PERFORM\_SORTING},
only those matches under the \texttt{Focus} specified are.}.

Another key implementation detail is that the \texttt{Match} constructor is guaranteed
to always refer to different slots in the sorting network, and that the first
slot is always less than (in terms of index) the second. This greatly simplifies
some logic such as keeping a sparse matrix of match results arranged \texttt{(Slot,
Slot) → Maybe Result}; since the values are always ordered and unequal, only an
upper triangle of such a matrix ever needs to be considered.

\newpage
\subsection{Final Definitions}
\label{sec:orgd959f6d}
I reproduce the final definition below:

\begin{lstlisting}[language=haskell,numbers=none,basicstyle=\ttfamily\small]
data Tournament :: Depth -> Type where
  One :: Match -> Tournament TOne
  Empty :: Tournament t
  -- | Modify a tournament's focus; that is, the slice of slots of the
  -- sorting network that it concerns
  SetFocus :: (Focus -> [Focus]) -> Tournament t -> Tournament (TMod t)
  -- | Overlay two tournaments, to describe running two sub-tournaments
  -- in parallel. The depth of the tournaments must be the same
  Overlay :: Tournament a -> Tournament a -> Tournament a
  -- | Sequence two tournaments one after the other. The resulting
  -- tournament has a depth 'TMany' which restricts what functions
  -- are able to manipulate it.
  Sequence :: (KnownDepth a, KnownDepth b)
           => Tournament a -> Tournament b -> Tournament TMany
  -- | Sort the inner tournament by some sorting method
  Sort :: SortMethod -> Tournament t -> Tournament t
  -- | Depend on the player count to produce an inner tournament
  ByPlayerCount :: (PlayerCount -> Tournament t) -> Tournament t
  -- | Depend on the current standings, at the outset of the current round, to
  -- run the tournament.
  ByStandings :: (Standings -> Tournament t) -> Tournament t
  -- | Lift a single round of a tournament into having a depth 'TMany'
  LiftTOne :: Tournament TOne -> Tournament TMany
  -- | Lift a modified round of a tournament into having a depth 'TMany'
  LiftTMod :: KnownDepth t => Tournament (TMod t) -> Tournament TMany

-- | The depth of a tournament. Since we only care about distinguishing single
-- rounds from sequences of rounds, that is the only information stored here.
data Depth = TOne | TMany | TMod Depth

-- | Reflect a type-level 'Depth' into a term-level value.
class Typeable d => KnownDepth (d :: Depth) where
  depthVal :: proxy d -> Depth

instance KnownDepth d => KnownDepth ('TMod d) where
  depthVal _ = depthVal (Proxy :: Proxy d)

instance KnownDepth 'TOne where
  depthVal _ = TOne

instance KnownDepth 'TMany where
  depthVal _ = TMany
\end{lstlisting}

Since there are numerous types in the sorting network representation that are
ultimately just \texttt{Int}, I introduce some \texttt{newtype} definitions to prevent the
accidentally mixing of, for instance, a sorting network slot, and a player.
These are named \texttt{Slot}, \texttt{Player}, and \texttt{RoundNo}. For convenience, each has the full
breadth of \texttt{Num} and \texttt{Integral} operations available that \texttt{Int} has.

\newpage
\subsection{The Journeyman eDSL: An accumulating continuation over \texttt{Tournament}}
\label{sec:org95bece4}

The final eDSL is defined as a continuation-based accumulation monad over the
original algebraic \texttt{Tournament} type. Since there are two main operations for
merging tournaments together (overlays and sequences), the builder is
parameterised by what depth it is at, which is used to provide the default merge
operation at that depth. That is, a builder of just matches within a round is a
\texttt{Round \textasciitilde{} Builder TOne}; a builder of rounds then is a \texttt{Steps \textasciitilde{} Builder TMany}.

The key convenience gained here is to be able to syntactically invert the
\texttt{ByPlayerCount} and \texttt{ByStandings} constructors; thus:

\begin{lstlisting}[language=haskell,numbers=none]
-- | Retrieve the current player count
getPlayerCount :: Merge t => Builder t () PlayerCount
getPlayerCount = _

-- | Retrieve the current standings
--
-- Warning: this operation will have the effect of segmenting
-- the tournament into two sections, as a tournament runner cannot
-- continue past this point without first having the standings.
getStandings :: Merge t => Builder t () Standings
getStandings = _
\end{lstlisting}

As the full API largely is a consequence of the primitives outlined above, they
will not be reproduced here. For API-centric documentation of the Journeyman
eDSL, see
\href{https://mikeplus64.github.io/journeyman/Tourney-Algebra-Builder.html}{ the online documentation}\footnote{\url{https://mikeplus64.github.io/journeyman/Tourney-Algebra-Builder.html}}. Most functions here simply add something to the current
\texttt{Tournament} accumulation, by whatever merging strategy (overlaying or
sequencing) is most appropriate.

\subsection{Tournament Interaction and Display}
\label{sec:org1437406}

In order to inspect and interact with tournaments, a terminal-based UI is built.
It enables the display and running of tournaments, and allows for inspection of
the core ``Virtual Machine'' that compiles and interprets tournaments into a
stream of commands, such as to easily identify the number of rounds/true depth
of a tournament, and its parallelisation (if any). Some screenshots of this tool
have already been included in this thesis.

\newpage

\section{Demonstration of the Journeyman eDSL}
\label{sec:org07594c6}

We shall now tour some tournament structure definitions I have created that use
the Journeyman eDSL.


\subsection{Round Robin and Group Stage Round Robins}
\label{sec:orgf1f1828}
The \emph{Circle Method} for round-robin scheduling is used here. The key point to
note here is the position of the \texttt{points} function, which requests sorting by
point awards for the matches within.

\begin{center}
\includegraphics[width=.9\linewidth]{./robin45.png}
\end{center}

\begin{lstlisting}[language=haskell,numbers=none]
roundRobin :: Steps () ()
roundRobin = points do
  count <- getPlayerCount
  let (!midpoint, !r) = count `quotRem` 2
  let !n = count + r
  mapM_
    (round_ . map match)
    [ foldAround midpoint (map Slot (0 : ((n - i) ..< n)
                                      ++ (1 ..< (n - i))))
    | i <- [0 .. n - 2]
    ]
\end{lstlisting}

More interestingly, we can use the \texttt{SetFocus}-based combinator \texttt{divideInto} to
run multiple parallel round-robin tournaments.

\begin{lstlisting}[language=haskell,numbers=none]
groupRoundRobin :: Int -> Steps () ()
groupRoundRobin numGroups = divideInto numGroups roundRobin
\end{lstlisting}

\newpage
\subsection{Insertion Sort, and \emph{I Can't Believe It Can Sort}}
\label{sec:org08e2046}

Three variations are provided here. The ``naiive'' variant of insertion sort here
follows the folk imperative definition of the insertion sort algorithm; but this
approach does not take advantage of the parallelisation available in a sorting
network. While it should be possible for the Journeyman tool to simplify the
graph of a tournament through an analysis of match dependencies, that is not
within the scope of work undertaken in this project.

A parallelised sorting network implementation of insertion sort is thus also
provided. Finally, the \emph{I Can't Believe It Can Sort} algorithm is provided. Note
that as expressed, the \emph{I Can't Believe It Can Sort} algorithm cannot be
parallelised; every match except for the initial one has a dependency on the
prior match's result. Possibly, we can improve this by creating a fully
saturated network that simply alternates between matching the odd-adjacent and
even-adjacent slots, but that idea is not pursued here.

\begin{figure}[htbp]
\centering
\includegraphics[width=.9\linewidth]{./insertsortnet.png}
\caption{Insertion sort of 8 players represented in Journeyman-UI.}
\end{figure}

\begin{lstlisting}[language=haskell,numbers=none,basicstyle=\ttfamily\small]
insertionSortNaiive :: Steps () ()
insertionSortNaiive = do
  n <- getPlayerCount
  i <- list (0 ..< Slot n)
  j <- list (i ..> 0)
  swaps (round_ (match (Match j (j - 1))))

insertionSortNetwork :: Steps () ()
insertionSortNetwork = do
  n <- Slot <$> getPlayerCount
  i <- list ([0 .. n - 2] ++ reverse [0 .. n - 3])
  swaps . round_ . asRound $ do
    let (m, r) = i `divMod` 2
    j <- list [0 .. m]
    match (Match (j * 2 + r) (j * 2 + 1 + r))

iCan'tBelieveItCanSort :: Steps () ()
iCan'tBelieveItCanSort = do
  n <- getPlayerCount
  i <- list (0 ..< Slot n)
  j <- list (0 ..< Slot n)
  when (i /= j) do
    round_ $ swaps $ match (Match i j)
\end{lstlisting}

\newpage
\subsection{Single Elimination}
\label{sec:orge2aa0d1}

Single Elimination tournaments turn out to be quite easy to represent in
Journeyman, especially when compared to traditional binary-tree approaches,
where the exact order of nodes/leaves has a determining effect on outcomes. The
key observation is that for the first round, with the simple arrangement of
players from most skilled to least skilled, we can use list operations to split
that list into two and fold the two sides together; thus making the most skilled
player meet the least skilled, and the second-most skilled to the second-least
skilled, and so on, until the two players at the middle level meet. Since this
operation is quite common in tournament construction, it is given the name
\texttt{foldAroundMidpoint}.

\begin{lstlisting}[language=haskell,numbers=none]
singleElimination :: Steps () ()
singleElimination = do
  count <- getPlayerCount
  let depth = log2 (2^ceil (log2 count))
  d <- list [depth, depth-1 .. 1]
  swaps (round_ (foldAroundMidpoint [0 .. 2^d - 1]))
\end{lstlisting}

Note that this tournament does output bye matches between non-existent slots if
the player count is not a power of 2. These matches are treated as automatic
wins for the valid player, which is a necessary common practice for elimination
tournaments that have a binary-tree structure.

\begin{center}
\includegraphics[width=.9\linewidth]{./singleelim8-2.png}
\end{center}

\newpage

\subsection{Double Elimination}
\label{sec:org4b87782}

\begin{figure}[htbp]
\centering
\includegraphics[width=.9\linewidth]{./kuachide.png}
\caption{An example Double-Elimination bracket hosted on \texttt{kuachi.gg} \autocite{aql51}}
\end{figure}

For Double Elimination it is worth first considering how an analogous sorting
network can be constructed. Obviously, the upper bracket can proceed identically
to a single elimination tournament, so we can definitely express it as some
overlay operation like \texttt{singleElimination +++ lowerBracket}. To elucidate what
structure is appropriate, I diagram the structure required for an 8-player
Double Elimination sorting network:

\begin{figure}[htbp]
\centering
\includegraphics[width=.9\linewidth]{./de-8-network.png}
\caption{For contrast, the same Double Elimination bracket expressed as a sorting network.}
\end{figure}

I use the result provided by \autocite{double-elim-structure-efficacy-manipulation}
to create a linking function.

\begin{lstlisting}[language=haskell,numbers=none,basicstyle=\ttfamily\small]
-- A direct implementation of Double Elimination tournaments.
--
-- The approach here is to:
-- 1. Construct a single elimination upper bracket
-- 2. Create an initial lower bracket round, from the lowers of the first round of
--    the single elimination bracket
-- 3. For each other upper bracket round:
--    3.1. Create a lower bracket round that accepts the losers from that round
--    3.2. Create a lower bracket round that plays off only LB players against
--         eachother.
--
-- Since we only depend on the rounds generated in the upper bracket, we can
-- parameterise that; so our "doubleElimination" function is generalisable to be
-- able to add an extra loser's bracket to _any_ tournament, although it is
-- likely to have strange results when given tournaments that are not in the
-- same shape as single-elimination.
--
-- Thus we can use this to create quadruple elimination brackets, but not
-- triple.
doubleElimination :: Steps () ()
doubleElimination = addLosersBracket singleElimination

addLosersBracket :: Steps () () -> Steps () ()
addLosersBracket original = do
  Right (ub1 :< ubs) <- inspect (ByRound Flat) original
  let lowerRound1 = foldAroundMidpoint (ub1 ^.. each . likelyLoser)
  swaps (round_ ub1)
  swaps (round_ lowerRound1)
  evaluatingStateT (lowerRound1 ^.. each . likelyWinner) do
    (i, upper) <- lift (list (V.indexed ubs))
    lastWinners <- get
    let shuffledLosers = linkFun i (upper ^.. each . likelyLoser)
    -- Accept new losing players from the upper bracket
    let acceptRound = zipWith Match lastWinners shuffledLosers
    -- Then perform a round of just lower bracket players being eliminated
    -- I.e., match up the winners of the round_ we just wrote
    let losersRound = foldAroundMidpoint (acceptRound ^.. each . likelyWinner)
    -- Finally, add these rounds, and store the winning players in losersRound
    -- for the next iteration
    round_ do
      swaps (toRound upper)
      swaps (toRound acceptRound)
    round_ (swaps (toRound losersRound))
    put (losersRound ^.. each . likelyWinner)

linkFun :: Int -> [a] -> [a]
linkFun size = foldr (.) id (replicate size linkFunSwap)

linkFunSwap :: [a] -> [a]
linkFunSwap l = drop h l ++ take h l
  where
    h = length l `div` 2
\end{lstlisting}

\newpage

\section{Future work and limitations}
\label{sec:orgfa959e5}

There is little way to verify that the tournament structure generated is the one
that is actually intended, other than to view it in the Journeyman-UI tool.
Additionally, analysis is currently limited; the UI tool will tell you how many
rounds a tournament has, but that is the extent that the information goes.

Indeed, an aim of this project originally was to create a Tournament ``Virtual
Machine'' (TVM) capable of statistical analysis of any arbitrary tournament
structure. While a TVM is developed and capable of evaluating tournaments,
including displaying their pure subset ahead of actually arriving at those
sections, it does not have any simulation capability. This is mainly owing to a
lack of time; in October I suffered a concussion and could not work on this
project for some time.

Implementing such a simulator should be a matter of connecting a TVM to a
simulator that repeatedly reads pending matches and evaluates them according to
a \(P_\text{win}\) function such as that provided by the Elo rating system. At the
conclusion of the tournament, compare the initial standings to the final
standings and count the number of inversions; this then forms the basis for a
metric of fairness known as rank preservation.

Additionally, we can test the robustness of a tournament against certain forms
of manipulation or error, such as mistakenly seeding a player too high or too
low. Again by counting the number of inversions at the end of the tournament and
comparing it to the number and extent of inversions in the beginning, we can
create a metric for reliability or how well the tournament actually ranks
players against each-other -- rather than simply replicating the input
standings.

\newpage

\printbibliography
\end{document}