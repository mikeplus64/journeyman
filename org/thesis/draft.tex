% Created 2023-11-19 Sun 10:07
% Intended LaTeX compiler: pdflatex
\documentclass[a4,11pt,twoside,final]{article}
\usepackage[utf8]{inputenc}
\usepackage[T1]{fontenc}
\usepackage{graphicx}
\usepackage{longtable}
\usepackage{wrapfig}
\usepackage{rotating}
\usepackage[normalem]{ulem}
\usepackage{amsmath}
\usepackage{amssymb}
\usepackage{capt-of}
\usepackage{hyperref}
\usepackage{geometry}
\newgeometry{bottom=2in,top=2in,inner=1in,outer=1in}
\usepackage{merriweather}
\usepackage{hyperref}
\usepackage{minted}
\usepackage{multicol}
\newcommand{\point}[1]{\noindent \textbf{#1}}
\usepackage{mathtools} % DeclarePairedDelimiter
\DeclarePairedDelimiter\ceil{\lceil}{\rceil}
\DeclarePairedDelimiter\floor{\lfloor}{\rfloor}
\author{Mike Ledger, supervised by Ranald Clouston, at Australian National University}
\date{November 20 2023}
\title{An Algebraic Functional Domain-Specific Language for Tournament Structures\\\medskip
\large Algebraic Graph Construction of Sorting-Network-like Structures}
\hypersetup{
 pdfauthor={Mike Ledger, supervised by Ranald Clouston, at Australian National University},
 pdftitle={An Algebraic Functional Domain-Specific Language for Tournament Structures},
 pdfkeywords={tournaments,edsl,haskell,fp,graphs,sorting,interactive,tui},
 pdfsubject={},
 pdfcreator={Emacs 29.1 (Org mode 9.7)}, 
 pdflang={English}}
\usepackage{calc}
\newlength{\cslhangindent}
\setlength{\cslhangindent}{1.5em}
\newlength{\csllabelsep}
\setlength{\csllabelsep}{0.6em}
\newlength{\csllabelwidth}
\setlength{\csllabelwidth}{0.45em * 4}
\newenvironment{cslbibliography}[2] % 1st arg. is hanging-indent, 2nd entry spacing.
 {% By default, paragraphs are not indented.
  \setlength{\parindent}{0pt}
  % Hanging indent is turned on when first argument is 1.
  \ifodd #1
  \let\oldpar\par
  \def\par{\hangindent=\cslhangindent\oldpar}
  \fi
  % Set entry spacing based on the second argument.
  \setlength{\parskip}{\parskip +  #2\baselineskip}
 }%
 {}
\newcommand{\cslblock}[1]{#1\hfill\break}
\newcommand{\cslleftmargin}[1]{\parbox[t]{\csllabelsep + \csllabelwidth}{#1}}
\newcommand{\cslrightinline}[1]
  {\parbox[t]{\linewidth - \csllabelsep - \csllabelwidth}{#1}\break}
\newcommand{\cslindent}[1]{\hspace{\cslhangindent}#1}
\newcommand{\cslbibitem}[2]
  {\leavevmode\vadjust pre{\hypertarget{citeproc_bib_item_#1}{}}#2}
\makeatletter
\newcommand{\cslcitation}[2]
 {\protect\hyper@linkstart{cite}{citeproc_bib_item_#1}#2\hyper@linkend}
\makeatother\begin{document}

\maketitle
\begin{abstract}
An Embedded Domain-Specific Language (commonly known as eDSL) is developed for
relatively easy expression and analysis of arbitrary tournament structures. The
core tournament structure type inspired by work on algebraic graph
representations in Haskell, and sorting networks, enabling composition in terms
of either interleaving or sequencing. The core design balances between allowing
arbitrary logic within a tournament structure and what static analysis can be
achieved given such a structure. Metrics such as length, fairness, resource use,
and reliability, are analysed on a variety of tournament structures. Similarity
between sorting algorithms - in particular sorting networks - is observed and
common tournament structures are contrasted with sorting networks to see how
viable they can be as tournament structures. A tool is also developed to perform
and visualise tournaments described within the eDSL.
\end{abstract}
\newpage

\newgeometry{margin=2cm}
\begin{multicols}{2}
\setcounter{tocdepth}{2}
\tableofcontents
\end{multicols}

\newpage
\newgeometry{top=2.5cm,bottom=2.5cm,left=4cm,right=2.5cm}

\section{Introduction}
\label{sec:orgf54d4f4}

Tournament structures are used to determine a ranking of players in a game, such
as a real-life sport or an eSport. Many different tournament structures are
employed in real games, for a variety of reasons; for instance length, fairness,
resources available, are all determining factors. The choice of what tournament
structure is most appropriate for a given set of circumstance is mostly an open
one, and many tournaments are ran as a composition of other tournament types.

There is then a possible contribution to be made in a software tool that would
help design and evaluate tournament structures. This project encapsulates an
effort to do that over the course of my undergraduate Advanced Computing Project
(COMP4560), by creating a Embedded Domain-Specific Language for tournament
description and evaluation, as a Haskell library. The software library developed
for this project is available online on \href{https://github.com/mikeplus64/journeyman}{GitHub}, which also hosts \href{https://mikeplus64.github.io/journeyman/}{API
documentation}.

\newpage

\section{Background}
\label{sec:org2af391e}

The economics of sports and eSports tournaments are also worth noting here.  The
social impact of this is out of scope for this project, but clearly with [XXX:
growth in Twitch/stream viewerships] these tournaments are becoming a
significant part of the zeitgeist.



\section{Old Background}
\label{sec:org6e4e683}

\section{Old Aims}
\label{sec:orgceed3e5}
\begin{enumerate}
\item To create a computer language that enables the description of common
tournament structures, and also to enable creation of new tournament
structures as compositions of existing ones.

\item To create an interpreter for this language that can evaluate tournament
descriptions; to be able to run the tournaments described by the language
defined in Goal 1.
\item To systematically evaluate the efficacy of different tournament structures in
terms of quantifiable properties such as fairness, length, repeat-avoidance,
and so on.
\end{enumerate}

These aims are underwritten by my own experience as a tournament organiser and
developer of a popular website that facilitates the running of eSport
tournaments; in running these tournaments, questions of fairness and the ideal
arrangement of tournament brackets are commonplace; validating existing methods
or otherwise creating newer and more effective ones would be a valuable
development here. \footnote{In my experience, the question of what tournament
structure is most effective comes up very frequently, and players often do feel
their placement in a tournament was unfair, or a player's seeding unduly caused
them to place higher or lower than they ``should'' have according to an assumed
skill level. I believe avoiding these scenarios would improve player
satisifaction and enjoyment.}

A key observation of this project is the existence of an equivalent sorting
network to many tournament structures. In particular, any tournament structure
whose rounds do not depend on the results gathered from prior rounds, and whose
matches can be expressed in Compare-and-Swap Matches (CASM), have an equivalent
sorting network construction. A hierarchy of match types is then constructed,
and it is shown that CASM is a specialisation of Points Matches (PM), from which
we can also construct tournaments based on ranking systems such as Elo
((cite:@elo)). These are common

This fact that a tournament can be constructed from any comparison-based sorting
algorithm (swapping comparison with a game/match\footnote{\textbf{Game} shall be used to
refer to a contest between two players from here on. The exact terminology is
not that important.}) is an important observation that I have made and informs
the aims of the project.

The research I will be conducting has these aims:

\begin{enumerate}
\item Creation of a domain-specific language (DSL) to describe tournament structures

\item Use of this DSL to describe common and novel tournament structures:
\begin{enumerate}
\item Round-robin
\item Elimination/knockout
\item Double/triple/\ldots{} elimination tournament
\item Swiss-style tournament
\item Arbitrary sorting network designs, such as insertion sort, bubble sort,
bitonic sort, etc.
\item Partial sorting algorithms or partitioning algorithms such as quickselect
or median of medians method
\end{enumerate}

\item Use of this DSL to describe compositions of created tournament structures:
\begin{enumerate}
\item Sequences of tournament structures (as in a ``multi-stage'' tournament)
\item Partitions of tournament structures by Divisions, where a each player in a
given Division \(D_1\) is expected to be of a higher skill than each player
in the next division \(D_2\), and so on.
\item Groups of tournaments, where each group is expected to have the same
average skill level as each other.
\end{enumerate}

\item Analysis of these tournament structures according to quantifiable metrics,
through simulation. The simulation can take place using generated skill
distributions (most likely through Elo \footnote{A common player ranking system,
designed originally for Chess, that allows for computing the probability of a
player winning against another.}), or by using real-world data such as Elo
distributions of Chess or Tennis players, which is publicly available data.

\begin{itemize}
\item Efficiency
\begin{itemize}
\item The time it takes to complete a tournament
\item The number of games that can be ran in parallel
\end{itemize}
\item Fairness
\begin{itemize}
\item repeat avoidance or not
\item Preservation of pre-determined skill levels
\item Maximum waiting/idle times for players between games
\end{itemize}
\item Robustness against manipulation - can strategically losing games improve a
player's final ranking? e.g., can strategically losing games in an early
round ensure an easier later tournament?
\item Robustness against inaccurate seeding (where the true skill level of a
player does not coincide with their seeding \footnote{The rank a player is
assigned at the outset of a tournament. This is especially important in
elimination-style tournaments. The worst-case scenario helps illustrate
why: If the player with the true highest skill and the player with the
second true highest skill are \uline{seeded} to play each other in the first round,
either one is guaranteed to be immediately eliminated.} - such as if they
simply have a ``bad day'', or their skill level was perceived as higher or
lower than it truly is by an event organiser.)
\item Ranking precision
\item Reliability - how repeatable is the result of a tournament?
\end{itemize}
\end{enumerate}

Given these aims, it is important to consider the existing literature on topics.
These are the topics I am looking for roughly. I use ``efficacy'' here to
encompass fairness, reliability, etc., as outlined above.

\begin{itemize}
\item Efficacy of tournament structures
\begin{itemize}
\item Simulations of tournament structures under real-world or generated skill
distributions
\item Improvements that can be made to existing tournament structures
\item Common pitfalls of existing tournament structures
\item Relative strengths or weaknesses of tournament structures
\end{itemize}
\item Studies on the efficacy of past real-world tournaments
\item Domain-specific language design and development
\item Existing software tooling that can help accomplish the aims above
\end{itemize}

Things out of scope of the research aims, but that may be useful to note,
include:
\begin{itemize}
\item Psychology of sports and eSports tournaments
\item Effect of different tournament structures on player performance
\item Player preference to different tournament structures
\item Perceived (by players) fairness of different tournament structures
\end{itemize}


\section{Literature Review}
\label{sec:orgce7efc2}

I proceed here to review XXX papers relevant to the above topics. Each section
is the title of a paper and links to a proper reference in the bibliography. It
is worth noting that this section overlaps with the notes I have written already
for the project in my GitHub repository for it, as it is my current COMP4560
project. \footnote{These notes are available \href{https://github.com/mikeplus64/journeyman/blob/main/docs/background/DSL\%20design.org}{online}, and so the similarity will
hopefully be noted by TurnItIn on submission of this literature review}.

Each paper has its own measure for what constitutes fairness, balance,
competitive development, etc., which are usually similar but not necessarily
identical. My research project therefore aims to provide a tool that makes it
easy to simulate tournaments and to measure such metrics in a unified fashion.

\subsection{``The structure, efficacy, and manipulation of double-elimination tournaments'' \cslcitation{1}{[1]}}
\label{sec:org9284aad}

This paper provides information about double-elimination tournaments in
particular, and in contrast to single-elimination tournaments. Several important
theorems are provided as well as statistical analysis performed on the efficacy
of tournaments.

Statistical analysis is performed to compare the reliability of
single-elimination tournaments to double-elimination ones, where
double-elimination is shown to be much more efficacious in allowing the most
skilled player to win than single-elimination. Simulations are performed using
chosen models for the probabilities of players winning against each other, rather
than on real-world data.

Manipulation of double-elimination tournaments is also considered, and an
interesting case study provided to demonstrate the need for tournaments that are
robust against manipulation: ``in the 2012 Olympics, four of the top badminton
teams were disqualified for trying to intentionally lose matches, causing an
uproar and angering fans. While the tournament structure used there was not a
DET, this demonstrates that players really will exploit poor tournament design
when possible.'' The importance of seeding in the outcome of elimination
tournaments is noted as well. Several theorems are provided on the complexity of
manipulation of a tournament by players.

Of note is that this paper provides a result that shows that a
double-elimination \uline{Link Function}. The Link Function is the algorithm that
chooses where in the lower bracket a player from the upper bracket should go
after a loss. The choice of Link Function is quite important in order to avoid
re-matching players who already faced each other in the upper bracket, for as long
as possible. It is shown that steps taken toward repeat avoidance need only be
done up to \(\log(R)\) with \(R\) the total number of rounds, and an algorithm is
provided as a suggested ``optimal'' Link Function using the provided primitives
(Swap and Reverse) for constructing link functions. I have implemented this
suggested algorithm in Rust in my tournament website \url{https://kuachi.gg}; the
implementation is \href{https://gitlab.com/\_mike/kuachicups/-/blob/master/server/src/db/tables/cup/stage/elim\_bracket/link\_fun.rs}{available online}.


\subsection{``Double-Elimination Tournaments: Counting and Calculating'' \cslcitation{2}{[2]}}
\label{sec:org1260384}

This paper provides broad information about the construction of Double
Elimination tournaments. The efficacy of ``unbalanced'' double-elimination
tournaments is considered in detail. A system for uniquely numbering
single-elimination tournaments is also provided, with extension then to number
double-elimination tournaments by the structure of the lower bracket as well as
the linking function used.

Statistical analysis is performed by using an assumed ``preference matrix'',
denoting the pairwise probabilities of one team winning a game against another.
This is an interesting approach that may be extremely difficult to compute for
larger tournaments (only 4 player tournaments are considered by preference
matrix), but offers several advantages over ``traditional'' ranking methods such
as Elo. In Elo, all players are assumed to have an absolute quantifiable skill
level, that satisfies transitivity; if player \(A\) is more skilled than player
\(B\), and player \(B\) is more skilled than player \(C\), then \(A\) must be more
skilled than player \(C\). A preference matrix approach allows for the fact that
some players may do particularly well or poorly against other players. It may be
possible to calculate a preference matrix from existing public data from
existing games, by assigning a secondary ranking to players by treating each
possible pair as its own separate game.

The larger double elimination tournament shown in this paper does not to have a
``balanced'' lower bracket. Convention in modern double-elimination tournaments is
that, to maximise fairness and minimise the number of rounds required, one
should alternate between rounds where players are from the lower bracket play
against each other, and where ``new'' players are added in to the lower bracket
from a round in the upper bracket. This is shown in
\cslcitation{1}{[1]}.


\subsection{``Simulating competitiveness and precision in a tournament structure: a reaper tournament'' \cslcitation{3}{[3]} and ``Reaper Tournament System'' \cslcitation{4}{[4]}}
\label{sec:org846ca2f}

I consider a pair of papers here sharing two of the same authors;
\cslcitation{4}{[4]} describes most of the results and \cslcitation{3}{[3]} develops the
knowledge of the Reaper tournament system further, and creates a similar (but
new) tournament structure called \uline{Reaper elimination}.

This paper proposes a novel tournament structure called a ``Reaper'' tournament.
It has several advantages to existing tournament structures, that are outlined
throughout the paper. This tournament structure is interesting as it is the sort
of structure that I would like to enable the creation and analysis of through
the DSL.

The structure of a Reaper tournament is not intuitive to me, but I repeat it
here in my own words in order to help my understanding of it that it operates as
an inverted single-elimination tournament initially, where only losers
``advance'', and from there a unique algorithm for repeatedly selecting and
eliminating the worst player is applied. This seems to have similarity to a
selection sorting algorithm. Because the Reaper tournament system is a complete
sorting algorithm, it has 100\% ranking precision.

The number of matches required in a Reaper tournament is not given a general
formula in the system, nor the number of rounds, which is a significant weakness
to its adoption as a tournament structure in practice - events need to happen
usually within known time constraints. Description of the Reaper tournament
system as a sorting network may help to elucidate its properties. For \(n=8\), the
Reaper tournament requires \(m\in[15,17]\) matches compared to \(m=14\) for double
elimination or \(m=28\) for a round-robin.

It is also shown that the \uline{stability progression}, measuring whether winning a
game is more desirable than losing, is preserved in the Reaper tournament
structure. It is never a desirable outcome to lose a match in the Reaper
tournament structure.

\subsection{Description of the Reaper tournament structure algorithm}
\label{sec:org5b9578b}
I reproduce in my own words the algorithm for the Reaper tournament structure
here.

Information:
\begin{itemize}
\item Each player has a \uline{respect list} of players who they have previously lost to.
This is updated every time a game occurs.
\item The tournament is assumed to be \(n=2^k\) in size; there must be a power-of-2
number of players.
\end{itemize}

Steps:
\begin{enumerate}
\item \uline{Reaper selection}: In Round 1, pairs of players are matched together, so that
every player is in a match. The losers in the round are then paired against
each other, and again, until a round where only a single player loses a match
(who lost all matches prior to this round), and they are eliminated from the
tournament. Let the winner of the final game in this step be the \uline{Reaper}.

This basically describes an ``inverted'' single elimination tournament - where
to proceed to the next round, you must \uline{lose} the current round. The ``winner''
(i.e., loser of all games) then of this inverted single elimination
tournament is the one who is actually eliminated from the tournament.

The question of what matching algorithm is used is left open by the authors
of the paper, but it is likely significant in determining the outcome of the
\uline{Reaper selection} stage.

\item \uline{Reaper candidates}: A \uline{candidate list} is created consisting of:
\begin{itemize}
\item If there are players who are not in a respect list, those players.
\item Otherwise, the players who are in the respect list of the Reaper.
\end{itemize}

The size of the candidate list then determines the next step:
\begin{itemize}
\item If \(> 1\), proceed to (3).
\item If \(= 1\), proceed to (4).
\item Otherwise (\(= 0\)), the tournament ends.
\end{itemize}

\item \uline{Candidates match}: The two best players play each other. Update the respect
lists accordingly and go back to step (2).

\item \uline{Reaper match}: The single player in the candidates list plays the Reaper. The
loser here is eliminated and is ranked above the previously eliminated
participant, while the winner is set to be the new Reaper.
\end{enumerate}


\subsection{Reaper Elimination}
\label{sec:orgcb1523f}
A new structure is proposed in the following paper \cslcitation{4}{[4]} that
develops the Reaper tournament structure to give it an upper bound on the number
of matches required, and a static tree structure. A visualisation of the Reaper
elimination tournament structure is provided in that demonstrates a static
structure to the tournament. Thus, it is a tournament that could be expressed as
a sorting network. It is shown that the number of matches required is \(O(N\log_2
N)\).

The second paper analyses two-stage tournament systems where a \uline{group stage}
precedes an \uline{elimination stage}. The \uline{group stage} has multiple groups of players in
each group, and a tournament structure such as round-robin (or Reaper), is
conducted within that group.

Various metrics are created to measure the efficacy of tournaments in practise
and in simulation. The key metrics are \uline{ENM}, meaning ``expected number of
matches'', \uline{ARW}, the ``average rank of the tournament winner'' (ideally, 1), and
\uline{RankCor} \(\in [0,1]\) where a value of 0 means the tournament had a completely
random result with respect to the player's ``true'' skills/rankings, and a value
of 1 means that the tournament perfectly preserved those a priori rankings.

Theoretical experiments for on 8 player tournaments are conducted that show the
excellent RankCor of the original Reaper tournament structure.
Double-elimination stages are also shown to have quite good RankCor (at this
size of tournament). Real-world tournament data is also used that demonstrates
the robustness of double-elimination tournaments in terms of \uline{RankCor}, with
Reaper tournaments also performing excellently for up to double the number of
matches required.


\subsection{``Quantifying the unfairness of the 2018 FIFO World Cup qualification'' \cslcitation{5}{[5]} and ``Risk of Collusion: Will Groups of 3 Ruin the FIFA World Cup?'' \cslcitation{6}{[6]}}
\label{sec:org3a875ac}

These papers look at real-world sports tournaments, namely the FIFA series of
soccer\footnote{Football?} tournaments. As these are huge events with massive prize
pools and carry great prestige for participating teams, nations, and hosts,
examination of these events for fairness criteria is important. These papers
demonstrate how real-world data can be used to examine and quantify fairness of
tournament structures.

It is shown in \cslcitation{5}{[5]} that the origin continent of a team
has an outsized effect on the likelihood of a team in qualifying into the FIFA
World Cup in 2018. The key takeaway is that a fixed draw rather than a random
draw for qualification would reduce unfairness. Unfairness is measured by ``[\ldots{}]
ranking the teams according to their Elo, and summing the differences of
qualifying probabilities that do not fall into line with this ranking''. This
unfairness metric may be useful in this research project in examination of the
fairness or not of arbitrary tournament structures.

In \cslcitation{6}{[6]}, the conditions required to aggravate the risk
of collusion between teams is examined. This can occur when two teams in a Group
stage are already guaranteed entry into the proceeding stage, but the result of
their match can adversely affect whether or not another team in that group makes
it through to the next stage or not. Examples of collusion are examined in
real-world games. Soccer seems especially susceptible to colluding outcomes as
draws are possible outcomes in the sport; teams may agree in advance to draw
against each other, and neither will lose face nor prestige, but both may then
be enabled to gain points required to proceed to the next stage of the
tournament. Such examples seem quite common. The risks of collusion are assigned
probabilities and examined in detail. Situations are examined where a team is
happy to lose by a small amount, and play to achieve that result.


\subsection{``Handling fairness issues in time-relaxed tournaments with availability constraints'' \cslcitation{7}{[7]}}
\label{sec:org9c7ef3c}

This paper examines computational complexity of time-relaxed tournament game
scheduling. That is, the problem of scheduling games where there is not a tight
deadline to complete the games, but there may be sporadic player and venue
availability. This situation frequently occurs during ``long format'' group stage
formats which are ran over weeks or months, where the scheduling of each game is
done by each player participating in that game together. However, this is out of
scope to the research aims of this project. The fairness measures proposed by
this paper also concern scheduling, which is outside of the scope of this
project.


\subsection{``The impossibility of a perfect tournament'' \cslcitation{8}{[8]}}
\label{sec:orgef3dfd7}

This paper provides an important result that shows that their constructed
\uline{fairness} and \uline{balance} metrics trade off against one-another, and elimination
tournaments cannot be constructed that maximise both metrics.

The \uline{fairness} metric here is that the sum of the ranks of winners of each match
must be maximised across the whole tournament. This is an interesting definition
that intuitively works quite well when the tournament structure is also
minimising the number of matches required - one could construct a degenerate
case tournament structure that maximises this sum, by, for example, matching 2
weak players repeatedly until the sum generated by the winners of those matches
must be greater than the sum generated by the winners of the other matches in
the tournament. Of course, that would not be an elimination tournament.

The \uline{balance} metric here is to minimise the difference in ranks between players
across all matches. By doing this, you create tournament structures that provide
as more information about players who are closely matched. In the single
elimination case, it is clear that maximising balance minimises fairness.
Maximising \uline{balance} can have the effect of increasing spectator interest, as
closer games are assumed to be more exciting to watch than ``blow-out'' games,
which I can validate from my own anecdotal experience.

The paper proves that a \uline{directed} tournament that contains any sub-tournament
where 4 unique players play 2 games in the same round, then the tournament
cannot minimise both the fairness and balance metrics. The terminology for
maximising and minimising fairness and balance is somewhat confusing in the
paper, as, for example, the section ``Tournaments that Maximize Both Fairness and
Balance'' discusses a tournament structure that in fact minimises fairness and
balance, and makes no mention of tournaments that maximise it. Taking the proof
at face value, that it shows that the balance and fairness metrics cannot be
\uline{minimised} (with the necessary conditions satisfied), this does not seem to
preclude the design of a tournament structure that \uline{maximises} the fairness and
balance metrics as the title of the paper would imply. This may just be my
limited understanding of this paper; a better reading may be required. In that
section, the paper provides an interesting 4-player tournament structure similar
to double-elimination, which does minimise (i.e., ``make bad'') the fairness and
balance metrics.

The author concludes that a perfect tournament design cannot be made because of
the inherent uncertainty of outcomes and player seeding; indeed, if perfect
ranking was already available at the outset, there would be little point to
running a tournament in the first place. The author also provides discussion on
the tournament outcomes and spectator interest; where players who play optimally
are perceived to be dull or unimaginative.

\subsection{``A new knockout tournament seeding method and its axiomatic justification'' \cslcitation{9}{[9]}}
\label{sec:orgb77ace2}

This paper demonstrates the determinativeness of seeding to single-elimination
tournament outcomes, and proposes an ``equal gap'' seeding method contrary to the
traditional ``slaughter seeding'' method, that, under a \uline{deterministic domain}
assumption, satisfies fairness, competitive integrity, and equal rank difference
axioms that are introduced. This assumption is roughly summarised as that for
any game \(m\) with players \(a\) and \(b\), the player with the highest seed (skill)
shall win, so the usefulness of equal gap seeding in practice is not completely
clear to me. It will be a research aim of my project to simulate the effect of
different seeding methods on tournament outcomes; equal-gap seeding provides a
plausible alternative to the standard elimination seeding method.

\subsection{``The efficacy of tournament designs'' \cslcitation{10}{[10]}}
\label{sec:orgb386b77}

This paper provides a great template for how statistical analysis and simulation
can be performed to demonstrate the superiority of particular tournament
structures over another in terms of given metrics. In particular, Swiss-style
tournaments are shown to be quite effective when compared to single or double
elimination tournaments, using generated Elo distributions as well as real-world
data from chess, soccer, and tennis.

Efficacy of tournaments is analysed in terms of ranking inversions exhibited at
the end of the tournament, which matches my intuition that tournament structures
can be expressed as sorting problems, and complements the aims of this project
quite well. A valuable result is that triple-elimination does not greatly
improve the efficacy of ranking players compared to double-elimination,
especially when accounting for the extra matches required.

Swiss-style tournaments are shown to be very effective at ranking players and
generally exhibit fewer inversions than any other format considered, for the
same number of matches - although Swiss-style tournaments use a matching
algorithm each round to determine who plays who, they are ran to a fixed number
of rounds, so they can be engineered to desired level of accuracy and matches.
Swiss-style tournaments are shown to be superior to single/double/triple
elimination and group stage tournaments.

The choice of matching algorithm here likely has the greatest effect on result,
which can be a result that my research project creates. Viewing Swiss-style
tournaments as partially-evaluated sorting networks enables this view. Indeed,
the comically titled sorting algorithm ``I Can't Believe It Can Sort''
\cslcitation{11}{[11]}, implemented as a mistaken version of insertion
or bubble sort, can be viewed as a Swiss-style tournament with a matching
algorithm that allows rematches, and ran to \(n\) rounds for \(n\) players; close
``elements'' - players - are repeatedly compared to create an accurate ranking of
those players.


\subsection{``Design Guidelines for Domain Specific Languages'' \cslcitation{12}{[12]}}
\label{sec:org6033e0f}

This paper provides a list of guidelines to follow for the design of DSLs, that
may be useful in this project as a DSL is a key artefact of it. In the case of
this paper, it is not so useful here to ``review'' it in the sense that other
literature is reviewed, but instead to respond to the guidelines posited, in
order to validate the aims of the DSL that is to be designed.

The guidelines identified, and my response to each with respect to this research
project's DSL, are:

\begin{enumerate}
\item \textbf{``Identify language uses early''}

The use of the language is identified in the introduction section of this
literature review; the design and implementation of novel tournament
structures, that maximise various metrics (ranking precision,
competitiveness, fairness, etc).

\item \textbf{``Ask questions''}

\begin{itemize}
\item \textbf{``Who is going to model in the DSL?''}
Myself, and tournament organisers who may find the software useful.

\item ``Who is going to review the models?''
Myself, and tournament organisers who may find the software useful.

\item ``When?'' During or near to project artefact completion, for use in creating
analysis that will be reported on in the final thesis paper.

\item ``Who is using the models for which purpose?''
\begin{itemize}
\item For myself: To identify and analyse the efficacy of various :tournament
structures
\item For other tournament organisers: Run novel tournament structures with
real players
\end{itemize}
\end{itemize}

\item ``Make your language consistent.''

The DSL should borrow existing semantics of existing tooling as much as
possible. I believe that semantics similar to \href{https://graphviz.org/doc/info/lang.html}{GraphViz DOT} will be
appropriate, with the addition of supporting mathematical operations, and
some looping or recursion constructs.

\item ``Decide carefully whether to use graphical or textual realization''

A textual representation will be the primary format for this DSL, due to the
extra effort required to design a visual system. However, it is noted that
existing tournament structures are often fully ``visual'' in nature; a
tournament organiser may elect to use a pen and paper to draw a tournament
structure and the progression of players through it. Therefore, visualisation
of the tournament structures \uline{after} creation from the DSL may be provide some
value.

\item ``Compose existing languages where possible'', and,
\item ``Reuse existing language definitions''
As above, the language design will take significant cues from GraphViz's DOT
format.

\item ``Reuse existing type systems''
This guideline raises the question of whether implementation of an eDSL may
be appropriate or not, as an eDSL can re-use the type system of its host
language, which may be quite valuable if the type system is fairly expressive
such as in a language like Haskell.
\end{enumerate}




\section{A Library for Tournament Design}
\label{sec:org81ca976}
\subsection{Goals}
\label{sec:orga4135e7}
\begin{enumerate}
\item Easy to use
\item Minimal number of primitives
\item Allow for analysis
-> Statically analyse the pure subsets of a tournament
\item Be able to express common tournament formats
\end{enumerate}

\subsection{Designs that were explored}
\label{sec:org99ba288}
\begin{itemize}
\item Indexing of triangular numbers
\begin{itemize}
\item A round-robin schedule has a triangular number of games. All rounds of all
tournaments are some subset of a round-robin schedule. So you could probably
express tournaments as an index into this domain; but how do you decide
those indices? You probably end up needing or wanting an eDSL anyway to get
those.
\end{itemize}
\item Match lists
\begin{itemize}
\item Just use \texttt{List (List Match)}. What tournaments can't be expresed like this?
\end{itemize}
\item Steps + Step builders
\item Limitations leading to final design
\end{itemize}

\subsection{A type for Tournaments}
\label{sec:org1de4929}
Algebraic graphs primer / credit to the idea

\subsection{Tournament primitives}
\label{sec:orgfd87dda}
\begin{itemize}
\item Type-encoded tournament depth
\item Overlays
\begin{itemize}
\item Meaning of at single round level
\item Meaning of at rounds level
\end{itemize}
\item Sequences
\begin{itemize}
\item Meaning of at rounds level
\end{itemize}
\item Focuses
\begin{itemize}
\item Meaning of at single round level
\item Meaning of at the rounds level
\end{itemize}
\item Sort methods
\begin{itemize}
\item Meaning of at single round level
\item Meaning of at rounds level
\end{itemize}
\end{itemize}

\subsection{A monad for constructing a Tournament}
\label{sec:orgac2bafe}
List-monad for accumulating tournaments
\begin{itemize}
\item ``Inverting'' the Tournament type with continuations, avoiding lambdas
\item Is it a law-abiding Monad?
\end{itemize}

\subsection{Compilation into sorting network-like structures}
\label{sec:orga712128}

\subsection{Optionally-dependent streams of matches}
\label{sec:org76342a0}
We have streams of matches that \uline{can} depend on an outside monad like IO to get
match results, but they can also run completely purely. They can be inspected
and ran without going through IO to say whether they are pure or not; the pure
``prefix'' of the tournament can then be returned, along with the impure ``suffix''.

\subsection{An instruction set for tournaments}
\label{sec:org821c6a6}

\subsection{Why are some tournaments not sorting networks?}
\label{sec:org4439004}
Normal vs point-award sorting networks. SE/DE are sorting networks, they have a
static description using matches where the winner takes the high position in the
sorting network. But round-robin schedules are also static descriptions, yet
they are not sorting networks; any ``upsets'' in the round-robin schedule would
cause it to perform a different tournament when expressed as a swapping sorting
network. I.e. it needs the players to only exchange positions after all matches
are complete. Since we are still able to inspect round robin tournaments
statically, whether or not a tournament is a normal sorting network is not a
good litmus test.

Call ``pure'' tournaments those which have a static description that do not depend
on the standings of the tournament for all matches to be output. Call ``impure''
ones tournaments that require standings to know all the matches. Since we can
add a primitive to change the sorting method, round-robin tournaments are ``pure''.

\subsection{Simulators of tournaments}
\label{sec:org8352686}
\subsection{Pure simulator}
\label{sec:org85da12f}
\subsection{Statistical simulator}
\label{sec:org4071c1a}
\subsection{IO simulator}
\label{sec:orge794dba}
\subsubsection{Piped}
\label{sec:orgbb788bd}
\subsubsection{Interactive}
\label{sec:org742989f}


\section{Tutorial}
\label{sec:org1e38428}
Each to have a short description, a visualisation of the generated sorting
network, and the code required to describe them.
\subsection{Specification of Common Formats}
\label{sec:orgd114a4e}
\subsection{Single and Double Elimination}
\label{sec:org39c0a19}
\begin{itemize}
\item ``Slaughter'' seeding
\end{itemize}
\subsection{Round Robin}
\label{sec:org52c1627}
\subsubsection{Round robin using standings}
\label{sec:org6d97e5d}
\subsubsection{Sorting methods}
\label{sec:orgd2ce802}
\subsection{Swiss-style}
\label{sec:org4164c85}

\subsection{Common Compositions}
\label{sec:org84eae7e}
\subsection{Groups of N}
\label{sec:org49bcbee}
\subsection{Groups of N, then combined back again}
\label{sec:org786b8fc}
\subsection{Accept only the top N players of the previous sub-tournament}
\label{sec:org113f327}

\subsection{Sorts}
\label{sec:org8f3d473}
\subsection{Insertion sort}
\label{sec:org2a59525}
\subsection{Bitonic sort}
\label{sec:orgca55aef}

\section{Analysis}
\label{sec:org7ce2db4}
\subsection{Rank preservation}
\label{sec:orgcb05687}
Assign elo to players and use that to simulate match results. At the end of the
tournament, compare expected (players sorted by elo) to actual. Use normal elo
distribution.

\section{Availability of this library}
\label{sec:orgab7144c}
Link to online documentation, package, source repository, etc.

\section{References}
\label{sec:org6504dc5}

\begin{cslbibliography}{0}{0}
\cslbibitem{1}{\cslleftmargin{[1]}\cslrightinline{I. Stanton and V. V. Williams, “The structure, efficacy, and manipulation of double-elimination tournaments,” \textit{Journal of quantitative analysis in sports}, vol. 0, no. 0, pp. 1–17, Jan. 2013, doi: \href{https://doi.org/10.1515/jqas-2012-0055}{10.1515/jqas-2012-0055}.}}

\cslbibitem{2}{\cslleftmargin{[2]}\cslrightinline{C. T. Edwahttps://doi.org/10.1515/jqas-2012-0055rds, “Double-elimination tournaments: Counting and calculating,” \textit{The american statistician}, vol. 50, no. 1, pp. 27–33, Feb. 1996, doi: \href{https://doi.org/10.1080/00031305.1996.10473538}{10.1080/00031305.1996.10473538}.}}

\cslbibitem{3}{\cslleftmargin{[3]}\cslrightinline{A. V. N. Dinh, N. P. H. Bao, M. N. A. Khalid, and H. Iida, “Simulating competitiveness and precision in a tournament structure: a reaper tournament system,” \textit{International journal of information technology}, vol. 12, no. 1, pp. 1–18, Nov. 2019, doi: \href{https://doi.org/10.1007/s41870-019-00397-5}{10.1007/s41870-019-00397-5}.}}

\cslbibitem{4}{\cslleftmargin{[4]}\cslrightinline{N. Pham, H. Bao, S. Xiong, and H. Iida, “Reaper tournament system,” 2017, pp. 16–33. doi: \href{https://doi.org/10.1007/978-3-319-73062-2_2}{10.1007/978-3-319-73062-2\_2}.}}

\cslbibitem{5}{\cslleftmargin{[5]}\cslrightinline{L. Csató, “Quantifying the unfairness of the 2018 fifa world cup qualification,” \textit{International journal of sports science \&amp; coaching}, vol. 18, no. 1, pp. 183–196, Apr. 2022, doi: \href{https://doi.org/10.1177/17479541211073455}{10.1177/17479541211073455}.}}

\cslbibitem{6}{\cslleftmargin{[6]}\cslrightinline{J. Guyon, “Risk of collusion: Will groups of 3 ruin the fifa world cup?,” \textit{Journal of sports analytics}, vol. 6, no. 4, pp. 259–279, Jan. 2021, doi: \href{https://doi.org/10.3233/jsa-200414}{10.3233/jsa-200414}.}}

\cslbibitem{7}{\cslleftmargin{[7]}\cslrightinline{D. Van Bulck and D. Goossens, “Handling fairness issues in time-relaxed tournaments with availability constraints,” \textit{Computers \&amp; operations research}, vol. 115, p. 104856, Mar. 2020, doi: \href{https://doi.org/10.1016/j.cor.2019.104856}{10.1016/j.cor.2019.104856}.}}

\cslbibitem{8}{\cslleftmargin{[8]}\cslrightinline{P. C. Placek, “The impossibility of a perfect tournament,” \textit{Entertainment computing}, vol. 45, p. 100540, Mar. 2023, doi: \href{https://doi.org/10.1016/j.entcom.2022.100540}{10.1016/j.entcom.2022.100540}.}}

\cslbibitem{9}{\cslleftmargin{[9]}\cslrightinline{A. Karpov, “A new knockout tournament seeding method and its axiomatic justification,” \textit{Operations research letters}, vol. 44, no. 6, pp. 706–711, Nov. 2016, doi: \href{https://doi.org/10.1016/j.orl.2016.09.003}{10.1016/j.orl.2016.09.003}.}}

\cslbibitem{10}{\cslleftmargin{[10]}\cslrightinline{B. R. Sziklai, P. Biró, and L. Csató, “The efficacy of tournament designs,” \textit{Computers \&amp; operations research}, vol. 144, p. 105821, Aug. 2022, doi: \href{https://doi.org/10.1016/j.cor.2022.105821}{10.1016/j.cor.2022.105821}.}}

\cslbibitem{11}{\cslleftmargin{[11]}\cslrightinline{S. P. Y. Fung, “Is this the simplest (and most surprising) sorting algorithm ever?” arXiv, 2021. doi: \href{https://doi.org/10.48550/ARXIV.2110.01111}{10.48550/ARXIV.2110.01111}.}}

\cslbibitem{12}{\cslleftmargin{[12]}\cslrightinline{G. Karsai, H. Krahn, C. Pinkernell, B. Rumpe, M. Schindler, and S. Völkel, “Design guidelines for domain specific languages.” arXiv, 2014. doi: \href{https://doi.org/10.48550/ARXIV.1409.2378}{10.48550/ARXIV.1409.2378}.}}

\end{cslbibliography}
\end{document}